%!TEX TS-program = pdflatex

\documentclass[12pt]{article}
\usepackage{amsfonts, amsmath, amssymb}
\usepackage{dcolumn, multirow}
\usepackage{setspace}
\usepackage{graphicx}
\usepackage{tabularx}
\usepackage{anysize, indentfirst, setspace}
\usepackage{verbatim, rotating, paralist}
\usepackage{latexsym}
\usepackage{amsthm}
\usepackage{parskip}
\usepackage{hyperref}
\usepackage{color}
\usepackage[right=2.5cm, left=2.5cm, top=3.5cm, bottom=3.5cm]{geometry} %right=, left=, top=, bottom=

\title{Julia Resources}
\author{Nick Eubank}
\date{\today}

\begin{document}
\maketitle

\subsection*{Everything from Today, Including This Doc}
\url{https://www.github.com/nickeubank/JuliaOverview}




\subsection*{Installation / Getting Julia}
\begin{itemize}
    \item You can get Julia at \url{www.julialang.org/downloads}.
    \item Interactive Development Environment: the most popular IDE is called Juno, and is provided through that Atom text editor by installing the \texttt{uber-juno} package. Note requires prior install of Julia from \url{www.julialang.org/downloads}. Detailed instructions: \url{https://github.com/JunoLab/uber-juno/blob/master/setup.md}
\end{itemize}


\subsection*{Forums / Where to Get Help}
\begin{itemize}
    \item Place for questions: \url{discourse.julialang.org}
    \item Where to subscribe for news on 1.0 releases: \url{https://discourse.julialang.org/c/announce}
\end{itemize}

\subsection*{Tutorials}
\begin{itemize}
    \item Great tutorials on \url{www.juliabox.com}. Just log in and go to Tutorials/Intro-To-Julia.
    \item For 2 hour video of Julia Computing instructor walking through these tutorials, go to \url{https://julialang.org/learning/} and select \emph{Intro to Julia}.
\end{itemize}

\subsection*{Cheat Sheets}
\begin{itemize}
    \item All the important Julia syntax: \url{https://juliadocs.github.io/Julia-Cheat-Sheet/}
    \item Side-by-side Julia, Python, Matlab syntax comparisons: \url{https://cheatsheets.quantecon.org/}
\end{itemize}


\subsection*{Julia on ACCRE}
\begin{itemize}
    \item Julia example scripts and installation: \url{https://github.com/accre/SLURM/tree/master/julia-job}
\end{itemize}

\subsection*{Commonly used / well-supported packages}
The Julia package community has developed into a set of families (e.g. JuliaStats, JuliaPlots, JuliaData, etc.). In general, the best packages are those found under one of these families (good packages created outside a family tend to get integrated). Here are some big ones:
\begin{itemize}
    \item \texttt{DataFrames.jl}: OK, this is a package not a family (the family is \href{https://github.com/JuliaData}{\underline{JuliaData}}), but it's probably the one you'll use most. It's basically \texttt{data.frames} from R -- it allows for tables whose columns are of different types. Details at \url{http://juliadata.github.io/DataFrames.jl/stable/}.
    \begin{itemize}
        \item Note DataFrames is currently a little slow with missing data -- this is one of the things the compiler update coming in 0.7 is meant to fix.
        \item The missing data type is currently a stand-alone library (\texttt{Missings.jl}), but it's getting integrated into the base library in 0.7.
        \item JuliaData also manages \texttt{CSV.jl}, \texttt{Feather.jl} for reading csvs and feather datasets in Julia as DataFrames.
    \end{itemize}
    \item \href{http://docs.juliaplots.org}{\underline{JuliaPlots}}: Manages the \texttt{Plot.jl} library, which provides a unified front end for plotting. It has one consistent interface, but can be used with lots of different back-ends, including \texttt{pyplot}, \texttt{plot.ly}, and \texttt{GR}. Details at \url{http://docs.juliaplots.org}.
    \item \href{https://github.com/JuliaIO/}{\underline{JuliaIO}}: manages most I/O libraries except for the two libraries in JuliaData noted above (\texttt{CSV.jl} and \texttt{Feather.jl}), including \texttt{HDF5.jl}, \texttt{JLD.jl} (for saving Julia objects as binaries), and \texttt{JSON.jl}.
    \item \href{https://www.juliaopt.org/}{\underline{JuliaOpt}}: set of packages for numerical optimization. \url{https://www.juliaopt.org/}
    \item \href{http://juliastats.github.io/}{\underline{JuliaStats}}: Family of packages that implement core statistics functionality (\url{https://github.com/JuliaStats/}). \texttt{Distributions.jl} for random number generators from lots of distributions, \texttt{StatsBase.jl} for basic (mostly-univariate) statistics, \texttt{GLM.jl} for linear models, and \texttt{MLBase.jl} for machine learning.
    \item \href{https://github.com/JuliaGraphs}{\underline{JuliaGraphs}}: \texttt{LightGraphs.jl} for lightweight graphs, \texttt{MetaGraphs.jl} for graphs with node and edge attributes.
\end{itemize}





\end{document}
